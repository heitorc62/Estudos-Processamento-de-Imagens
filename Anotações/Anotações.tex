\documentclass[a4paper, 12pt]{article}

\usepackage[portuges]{babel}
\usepackage[utf8]{inputenc}
\usepackage{amsmath}
\usepackage{indentfirst}
\usepackage{graphicx}
\setlength{\jot}{8pt}
\usepackage[colorinlistoftodos]{todonotes}


\begin{document}

\begin{titlepage}
	\begin{center}
		\huge{Anotações sobre a Iniciação Científica}

		\vspace{30pt}
		
	\end{center}
	
	\begin{flushleft}
		%\begin{tabbing}
		%	Alunos\qquad\qquad\= \>Heitor Barroso Cavalcante - 12566101\\\\
		%	Professora\> Nina S. T. Hirata \\
		%
	    %\end{tabbing}
		Aluno\qquad\qquad Heitor Barroso Cavalcante - 12566101\\
		\vspace{10pt}
		Professora \quad Nina S. T. Hirata \\
		  
	\end{flushleft}
	\vspace{30pt}
	\begin{center}
		21 de Abril de 2022
	\end{center}
	\tableofcontents
\end{titlepage}
%%%%%%%%%%%%%%%%%%%%%%%%%%%%%%%%%%%%%%%%%%%%%%%%%%%%%%%%%%%\\
\thispagestyle{empty}

\newpage
\pagenumbering{arabic}

%%%%%%%%%%%%%%%%%%%%%%%%%%%%%%%%%%%%%%%%%%%%%%%%%%%%%%%%%
%%%%%%%%%%%%%%%%%%%%%%%%%%%%%%%%%%%%%%%%%%%%%%%%%%%%
\section{Processamento de Imagens Digitais}
\subsection{Introdução}
O processamento de imagens é uma subárea da disciplina de ``Processamento de Sinais", e essa subárea pode ser dividida em processamento digital ou analógico de imagens.
\begin{itemize}
    \item Processamento Analógico de Imagens:\\
    As imagens são manipuladas através da variação de sinais elétricos, o maior exemplo deste tipo de processamento é a imagem de televisão.
    \item Processamento Digital de Imagens:\\
    Consiste em desenvolver sistemas digitais para manipular imagens.
\end{itemize}

Nesse sentido, imagens podem ser compreendidas como um sinal bidimensional e pode ser definida como uma função \(f(x,y)\) tal que \((x,y)\) são as coordenadas de um pixel e \(f(x,y\) é o valor deste pixel. Assim, figuras visualizadas em um computador, por exemplo são matrizes de inteiros que variam de 0 a 255. As dimensões da imagem são as dimensões da matriz.
\\

Um sinal, pode ser definido como qualquer quantidade física medida através do tempo, do espaço ou qualquer outra dimensão. Logo, uma imagem digital pode ser caracterizada como um sinal de duas dimensões.
\\

A formação de uma imagem digital parte de um processo físico. A luz refletida pelos objetos fotografados é medida por diversos sensores e uma tensão contínua é gerada de acordo com a quantidade de luz captada pelos sensores. Agora, esse sinal analógico deve ser convertido para um digital. Para isso, amostragem e quantização são utilizadas para gerar a matriz de números que forma a imagem digital.
\\

Visão Computacional consiste no desenvolvimento de um sistema que consiga, a partir de uma imagem, gerar uma saída contendo informações sobre a imagem. Por exemplo, sistemas com reconhecimento facial.
\\

Por fim, resumindo, o processamento de sinais é muito importante para
o processamento de imagens. Sensores captam a luz do mundo físico, resultando em um sinal bidimensional que, após ser processado, forma uma imagem digital. Aí então esta imagem é manipulada através do processamento digital de imagens.

\end{document}