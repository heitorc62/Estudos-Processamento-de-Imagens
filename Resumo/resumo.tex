\documentclass{article}

% Language setting
% Replace `english' with e.g. `spanish' to change the document language
\usepackage[portuguese]{babel}
\usepackage{indentfirst}

% Set page size and margins
% Replace `letterpaper' with `a4paper' for UK/EU standard size
\usepackage[letterpaper,top=2cm,bottom=2cm,left=3cm,right=3cm,marginparwidth=1.75cm]{geometry}

% Useful packages
\usepackage{amsmath}
\usepackage{graphicx}
\usepackage[colorlinks=true, allcolors=blue]{hyperref}

\title{Resumo do Projeto de Iniciação Científica}
\author{
  Aluno: Heitor Barroso Cavalcante\\
  \\
  Orientadora: Nina Sumiko Tomita Hirata\\
}


\begin{document}
\maketitle

\section{Descrição}
Em um primeiro momento, procurei a professora Nina Hirata do Departamento de Ciência da Computação do IME-USP
 para que ela me orientasse em um projeto de iniciação científica para estudar sobre machine learning em um
contexto de reconhecimento de padrões em plantas. Isso seria feito utilizando visão computacional e, como
sabia que essa era sua área de pesquisa, pensei que a professora Nina seria perfeita para me orientar. 
Assim, para que eu pudesse alcançar esse objetivo de pesquisa, decidimos começar a iniciação científica 
desenvolvendo tópicos mais básicos e, gradativamente, evoluir na complexidade dos temas abordados.
\\

Nesse sentido, a professora Nina sugeriu que seguíssemos um roteiro de estudos que cobrisse, respectivamente, 
os seguintes tópicos:

\subsection{Fundamentos de Processamento de Imagens}
Em um momento inicial do projeto, para que me familiarizasse com o processamento de imagens, começamos por esse
tópico. O objeto de estudo consistia em um conjunto de tutoriais sobre processamento de imagens digitais em 
paralelo à atividades práticas de processamento e manipulação de imagens utilizando Python, mais especificamente, 
a biblioteca Numpy.


\section{Bibliografia}



\bibliographystyle{alpha}

\end{document}